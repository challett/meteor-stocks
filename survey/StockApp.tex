\documentclass{sigchi}

% Use this command to override the default ACM copyright statement (e.g. for preprints). 
% Consult the conference website for the camera-ready copyright statement.


%% EXAMPLE BEGIN -- HOW TO OVERRIDE THE DEFAULT COPYRIGHT STRIP -- (July 22, 2013 - Paul Baumann)
% \toappear{Permission to make digital or hard copies of all or part of this work for personal or classroom use is 	granted without fee provided that copies are not made or distributed for profit or commercial advantage and that copies bear this notice and the full citation on the first page. Copyrights for components of this work owned by others than ACM must be honored. Abstracting with credit is permitted. To copy otherwise, or republish, to post on servers or to redistribute to lists, requires prior specific permission and/or a fee. Request permissions from permissions@acm.org. \\
% {\emph{CHI'14}}, April 26--May 1, 2014, Toronto, Canada. \\
% Copyright \copyright~2014 ACM ISBN/14/04...\$15.00. \\
% DOI string from ACM form confirmation}
%% EXAMPLE END -- HOW TO OVERRIDE THE DEFAULT COPYRIGHT STRIP -- (July 22, 2013 - Paul Baumann)


% Arabic page numbers for submission. 
% Remove this line to eliminate page numbers for the camera ready copy
% \pagenumbering{arabic}


% Load basic packages
\usepackage{balance}  % to better equalize the last page
\usepackage{graphics} % for EPS, load graphicx instead
\usepackage{times}    % comment if you want LaTeX's default font
\usepackage{url}      % llt: nicely formatted URLs

% llt: Define a global style for URLs, rather that the default one
\makeatletter
\def\url@leostyle{%
  \@ifundefined{selectfont}{\def\UrlFont{\sf}}{\def\UrlFont{\small\bf\ttfamily}}}
\makeatother
\urlstyle{leo}


% To make various LaTeX processors do the right thing with page size.
\def\pprw{8.5in}
\def\pprh{11in}
\special{papersize=\pprw,\pprh}
\setlength{\paperwidth}{\pprw}
\setlength{\paperheight}{\pprh}
\setlength{\pdfpagewidth}{\pprw}
\setlength{\pdfpageheight}{\pprh}

% Make sure hyperref comes last of your loaded packages, 
% to give it a fighting chance of not being over-written, 
% since its job is to redefine many LaTeX commands.
\usepackage[pdftex]{hyperref}
\hypersetup{
pdftitle={Proposal and Survey for a Stock Application},
pdfauthor={Jadees Anton, Connor Hallett, Spencer Lee, Nicolas Lelievre},
pdfkeywords={},
bookmarksnumbered,
pdfstartview={FitH},
colorlinks,
citecolor=black,
filecolor=black,
linkcolor=black,
urlcolor=black,
breaklinks=true,
}

% create a shortcut to typeset table headings
\newcommand\tabhead[1]{\small\textbf{#1}}


% End of preamble. Here it comes the document.
\begin{document}

\title{Proposal and Survey for a Stock Application}
\numberofauthors{4}
\author{
	\alignauthor Jadees Anton\\
		\email{antonj@mcmaster.ca}\\
	\alignauthor Connor Hallett\\
		\email{hallec3@mcmaster.ca}\\
	\alignauthor Spencer Lee\\
		\email{leese5@mcmaster.ca}\\
	\alignauthor Nicolas Lelievre\\
		\email{lelievnm@mcmaster.ca}
}

\maketitle

\begin{abstract}
Be sure to also indicate if you are submitting a DESIGN or RESEARCH project in the ``Abstract'' section of
your submission.
\end{abstract}


\section{Overview}
overview of the topic/software/device of interest – tell me why you want to develop such a system

\section{Design Improvements}
idea of what will be unique/better/improved in your system (versus what already exists – this will be
informed by your software survey)



\section{Critique 1}
\subsection{Description}
Brief description of the application.

\subsection{User Goals}
Identify the high-level goals of the user of such a system – what is common to this type of
application?

\subsection{User Tasks}
Identifying tasks the user must perform to achieve their goals.

\subsection{Critique}
How it's bad, bad, bad, good.



\section{Critique 2}
\subsection{Description}
Brief description of the application.

\subsection{User Goals}
Identify the high-level goals of the user of such a system – what is common to this type of
application?

\subsection{User Tasks}
Identifying tasks the user must perform to achieve their goals.

\subsection{Critique}
How it's bad, bad, bad, good.



\section{HTC Stocks}
\subsection{Description}
HTC Stocks is a mobile Android application that comes installed on all HTC devices.  This app allows the user to add stocks to their list of bookmarked stocks, view basic information about all stocks that they have bookmarked at the same time, and view detailed information about one stock at a time.  The detailed stock view includes a graph of price history, high and low prices, and information about the values at open and close for the stock on the current day.  The application is powered by Yahoo Finance.

%\subsection{User Goals}
%A user of this application will want to quickly view information about a select collection of stocks.  
%They will want to see basic information about all of the stocks that they follow, and detailed information 
%about specific stocks.  The user will also want to maintain their list of favorite stocks, so that the information
%that is displayed to them is the information that they want. 
%finding information about generals stocks and monitoring favorite stocks

\subsection{User Tasks}
For a user to find information about stocks they may be interested in through this application, they first need to add the stock in question to their favorites.  Once this is done, the stock will appear in the users list of favorite stocks, where they can tap on it to see detailed information.\\
To view general information about all of the stocks that a user is following, all that the user needs to do is open the app.  On the home view a list of all of the stocks that a user is currently following is displayed along with the current price of the stock, along with the absolute and relative changes in the stock's price over the trading day.

\subsection{Critique}
This application falls short when it comes to viewing information about stocks that a user may be interested in.  For a user to see any information about a stock, it must be added to their list of favorite stocks.  As a result, the user may end up with a favorites list that is cluttered with stocks that they only wanted to view once, which makes seeing information about their regular stocks more difficult.  This application does a good job however in displaying quick information to the user about their followed stocks.  If the user has maintained an accurate list of their favorite stocks, they will see the majority of the information they need as soon as they open the application.



\section{Critique 4}
\subsection{Description}
Brief description of the application.

\subsection{User Goals}
Identify the high-level goals of the user of such a system – what is common to this type of
application?

\subsection{User Tasks}
Identifying tasks the user must perform to achieve their goals.

\subsection{Critique}
How it's bad, bad, bad, good.









\section{Conclusion}
Paper conclusion (if we have one).




% Balancing columns in a ref list is a bit of a pain because you
% either use a hack like flushend or balance, or manually insert
% a column break.  http://www.tex.ac.uk/cgi-bin/texfaq2html?label=balance
% multicols doesn't work because we're already in two-column mode,
% and flushend isn't awesome, so I choose balance.  See this
% for more info: http://cs.brown.edu/system/software/latex/doc/balance.pdf
%
% Note that in a perfect world balance wants to be in the first
% column of the last page.
%
% If balance doesn't work for you, you can remove that and
% hard-code a column break into the bbl file right before you
% submit:
%
% http://stackoverflow.com/questions/2149854/how-to-manually-equalize-columns-
% in-an-ieee-paper-if-using-bibtex
%
% Or, just remove \balance and give up on balancing the last page.
%
\balance
% REFERENCES FORMAT
% References must be the same font size as other body text.

\bibliographystyle{acm-sigchi}
\bibliography{sample}
\end{document}
